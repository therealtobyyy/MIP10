% Metódy inžinierskej práce

\documentclass[10pt,twoside,slovak,a4paper]{article}

\usepackage[slovak]{babel}
%\usepackage[T1]{fontenc}
\usepackage[IL2]{fontenc} % lepšia sadzba písmena Ľ než v T1
\usepackage[utf8]{inputenc}
\usepackage{graphicx}
\usepackage{url} % príkaz \url na formátovanie URL
\usepackage{hyperref} % odkazy v texte budú aktívne (pri niektorých triedach dokumentov spôsobuje posun textu)

\usepackage{cite}
%\usepackage{times}

\pagestyle{headings}

\title{Virtuálna realita a zdravie\thanks{Semestrálny projekt v predmete Metódy inžinierskej práce, ak. rok 2015/16, vedenie: I Ing. Vladimír Mlynarovič PhD}} % meno a priezvisko vyučujúceho na cvičeniach

\author{Tobias Kopernický\\[2pt]
	{\small Slovenská technická univerzita v Bratislave}\\
	{\small Fakulta informatiky a informačných technológií}\\
	{\small \texttt{xkopernicky@stuba.sk}}
	}

\date{\small 18. október 2022} % upravte



\begin{document}

\maketitle

\begin{abstract}
Článok sa zaoberá virtuálnou realitou a jej vývojom za posledné roky, spolu s jej vplyvom na fyzické a psychické zdravie ľudí. Počas covidu boli klasické fitká zavreté, a ľudia boli silno obmedzený s formou ich cvičenia. Virtuálna realita priniesla možnosť vysokokvalitného tréningu z pohodlia domova, a to vo forme fitness hier, ako napríklad FitXR. FitXR sa skladá primárne z vysoko intenzívnych tréningov, ktoré s pomocou hudby a interaktívneho prostredia zabezpečujú cvičenie, ktoré dá človeku zabrať. Spolu z ich vlastnými virtuálnymi trénermi máte zabezpečený optimálny workout prostredníctvom hier simulujúce box alebo tanec. Využitie virtuálnej reality sa taktiež začalo prelínať do prostredia zdravotníctva, ako nástroj ma pomoc ošetrenia rôznych fyzických a psychických chorôb.
\end{abstract}


\section{Úvod}

Virtuálna realita je rýchlo sa vyvíjajúca technológia ktorá má za cieľom pohltiť užívateľa v 3D priestore. Využitie VR sa za posledné roky viac a viac dostáva do oblasti medicíny a zdravotníctva na podporu liečby, taktiež aj zlepšovanie vedomostí študentov a obohatenie ich vedomostí.



\section{Zoznámenie sa s VR} \label{Zoznamenie}

Virtuálna reality alebo VR je simulované prostredie procesované softvérom a hardvérom, ktoré používateľ pohltí v digitálnom prostredí, ktoré sa snaží do istej miery replikovať prvky reality. Toto pohltenie je dosiahnuté vďaka implementácii rôznych zariadení, ktoré stimulujú naše zmysli v umelom prostredí. Zmysli ako zrak a sluch sú využité najviac vďaka dvom ku jednotlivým očiam a 360 priestorovým audiom.

\section{Ako funguje VR?} \label{Funguje}
 
Virtuálna Reality simuluje 3 dimenzionálny priestor kde všetky zorné uhly sú pokryté simuláciou, pričom vo viac vyvinutých prostrediach má používateľ možnosť aj sa hýbať v rozsahu prostredia a aj sním manipulovať vďaka doplnkovými perifériami.

\section{Čo potrebujeme na VR?} \label{Devices}

\begin{figure*}[tbh]
\centering
%\includegraphics[scale=1.0]{diagram.pdf}
Aj text môže byť prezentovaný ako obrázok. Stane sa z neho označný plávajúci objekt. Po vytvorení diagramu zrušte znak \texttt{\%} pred príkazom \verb|\includegraphics| označte tento riadok ako komentár (tiež pomocou znaku \texttt{\%}).
\caption{Rozhodujúci argument.}
\label{f:rozhod}
\end{figure*}



\section{Iná časť} \label{ina}

Základným problémom je teda\ldots{} Najprv sa pozrieme na nejaké vysvetlenie (časť~\ref{ina:nejake}), a potom na ešte nejaké (časť~\ref{ina:nejake}).\footnote{Niekedy môžete potrebovať aj poznámku pod čiarou.}

Môže sa zdať, že problém vlastne nejestvuje\cite{Coplien:MPD}, ale bolo dokázané, že to tak nie je~\cite{Czarnecki:Staged, Czarnecki:Progress}. Napriek tomu, aj dnes na webe narazíme na všelijaké pochybné názory\cite{PLP-Framework}. Dôležité veci možno \emph{zdôrazniť kurzívou}.


\subsection{Nejaké vysvetlenie} \label{ina:nejake}

Niekedy treba uviesť zoznam:

\begin{itemize}
\item jedna vec
\item druhá vec
	\begin{itemize}
	\item x
	\item y
	\end{itemize}
\end{itemize}

Ten istý zoznam, len číslovaný:

\begin{enumerate}
\item jedna vec
\item druhá vec
	\begin{enumerate}
	\item x
	\item y
	\end{enumerate}
\end{enumerate}


\subsection{Ešte nejaké vysvetlenie} \label{ina:este}

\paragraph{Veľmi dôležitá poznámka.}
Niekedy je potrebné nadpisom označiť odsek. Text pokračuje hneď za nadpisom.



\section{Dôležitá časť} \label{dolezita}




\section{Ešte dôležitejšia časť} \label{dolezitejsia}




\section{Záver} \label{zaver} % prípadne iný variant názvu



%\acknowledgement{Ak niekomu chcete poďakovať\ldots}


% týmto sa generuje zoznam literatúry z obsahu súboru literatura.bib podľa toho, na čo sa v článku odkazujete
\bibliography{literatura}
\bibliographystyle{alpha} % prípadne alpha, abbrv alebo hociktorý iný
\end{document}
